% This is "sig-alternate.tex" V2.1 April 2013
% This file should be compiled with V2.5 of "sig-alternate.cls" May 2012
%
% This example file demonstrates the use of the 'sig-alternate.cls'
% V2.5 LaTeX2e document class file. It is for those submitting
% articles to ACM Conference Proceedings WHO DO NOT WISH TO
% STRICTLY ADHERE TO THE SIGS (PUBS-BOARD-ENDORSED) STYLE.
% The 'sig-alternate.cls' file will produce a similar-looking,
% albeit, 'tighter' paper resulting in, invariably, fewer pages.
%
% ----------------------------------------------------------------------------------------------------------------
% This .tex file (and associated .cls V2.5) produces:
%       1) The Permission Statement
%       2) The Conference (location) Info information
%       3) The Copyright Line with ACM data
%       4) NO page numbers
%
% as against the acm_proc_article-sp.cls file which
% DOES NOT produce 1) thru' 3) above.
%
% Using 'sig-alternate.cls' you have control, however, from within
% the source .tex file, over both the CopyrightYear
% (defaulted to 200X) and the ACM Copyright Data
% (defaulted to X-XXXXX-XX-X/XX/XX).
% e.g.
% \CopyrightYear{2007} will cause 2007 to appear in the copyright line.
% \crdata{0-12345-67-8/90/12} will cause 0-12345-67-8/90/12 to appear in the copyright line.
%
% ---------------------------------------------------------------------------------------------------------------
% This .tex source is an example which *does* use
% the .bib file (from which the .bbl file % is produced).
% REMEMBER HOWEVER: After having produced the .bbl file,
% and prior to final submission, you *NEED* to 'insert'
% your .bbl file into your source .tex file so as to provide
% ONE 'self-contained' source file.
%
% ================= IF YOU HAVE QUESTIONS =======================
% Questions regarding the SIGS styles, SIGS policies and
% procedures, Conferences etc. should be sent to
% Adrienne Griscti (griscti@acm.org)
%
% Technical questions _only_ to
% Gerald Murray (murray@hq.acm.org)
% ===============================================================
%
% For tracking purposes - this is V2.0 - May 2012

\documentclass{sig-alternate-05-2015}


\begin{document}

% Copyright
%\setcopyright{acmcopyright}
%\setcopyright{acmlicensed}
%\setcopyright{rightsretained}
%\setcopyright{usgov}
%\setcopyright{usgovmixed}
%\setcopyright{cagov}
%\setcopyright{cagovmixed}


% DOI
%\doi{10.475/123_4}

% ISBN
%\isbn{123-4567-24-567/08/06}

%Conference
%\conferenceinfo{PLDI '13}{June 16--19, 2013, Seattle, WA, USA}

%\acmPrice{\$15.00}

%
% --- Author Metadata here ---
%\conferenceinfo{WOODSTOCK}{'97 El Paso, Texas USA}
%\CopyrightYear{2007} % Allows default copyright year (20XX) to be over-ridden - IF NEED BE.
%\crdata{0-12345-67-8/90/01}  % Allows default copyright data (0-89791-88-6/97/05) to be over-ridden - IF NEED BE.
% --- End of Author Metadata ---

\title{Software Testing, Validation and Verification Class Project Proposal}
%\subtitle{[Extended Abstract]
%\titlenote{A full version of this paper is available as
%\textit{Author's Guide to Preparing ACM SIG Proceedings Using
%\LaTeX$2_\epsilon$\ and BibTeX} at
%\texttt{www.acm.org/eaddress.htm}}}
%
% You need the command \numberofauthors to handle the 'placement
% and alignment' of the authors beneath the title.
%
% For aesthetic reasons, we recommend 'three authors at a time'
% i.e. three 'name/affiliation blocks' be placed beneath the title.
%
% NOTE: You are NOT restricted in how many 'rows' of
% "name/affiliations" may appear. We just ask that you restrict
% the number of 'columns' to three.
%
% Because of the available 'opening page real-estate'
% we ask you to refrain from putting more than six authors
% (two rows with three columns) beneath the article title.
% More than six makes the first-page appear very cluttered indeed.
%
% Use the \alignauthor commands to handle the names
% and affiliations for an 'aesthetic maximum' of six authors.
% Add names, affiliations, addresses for
% the seventh etc. author(s) as the argument for the
% \additionalauthors command.
% These 'additional authors' will be output/set for you
% without further effort on your part as the last section in
% the body of your article BEFORE References or any Appendices.

\numberofauthors{2} %  in this sample file, there are a *total*
% of EIGHT authors. SIX appear on the 'first-page' (for formatting
% reasons) and the remaining two appear in the \additionalauthors section.
%
\author{
% You can go ahead and credit any number of authors here,
% e.g. one 'row of three' or two rows (consisting of one row of three
% and a second row of one, two or three).
%
% The command \alignauthor (no curly braces needed) should
% precede each author name, affiliation/snail-mail address and
% e-mail address. Additionally, tag each line of
% affiliation/address with \affaddr, and tag the
% e-mail address with \email.
%
% 1st. author
\alignauthor
Juan Manuel Florez Fandino \\ %\titlenote{Dr.~Trovato insisted his name be first.}\\
       \affaddr{University of Texas at Dallas}\\
       \affaddr{800 W. Campbell Road}\\
       \affaddr{Richardson, TX}\\
       \email{jflorez@utdallas.edu}
% 2nd. author
\alignauthor
Raul Quinonez Tirado \\ 
       \affaddr{University of Texas at Dallas}\\
       \affaddr{800 W. Campbell Road}\\
       \affaddr{Richardson, TX}\\
       \email{Rxq100020@utdallas.edu}
%% 3rd. author
%\alignauthor Lars Th{\o}rv{\"a}ld\titlenote{This author is the
%one who did all the really hard work.}\\
%       \affaddr{The Th{\o}rv{\"a}ld Group}\\
%       \affaddr{1 Th{\o}rv{\"a}ld Circle}\\
%       \affaddr{Hekla, Iceland}\\
%       \email{larst@affiliation.org}
%\and  % use '\and' if you need 'another row' of author names
%% 4th. author
%\alignauthor Lawrence P. Leipuner\\
%       \affaddr{Brookhaven Laboratories}\\
%       \affaddr{Brookhaven National Lab}\\
%       \affaddr{P.O. Box 5000}\\
%       \email{lleipuner@researchlabs.org}
%% 5th. author
%\alignauthor Sean Fogarty\\
%       \affaddr{NASA Ames Research Center}\\
%       \affaddr{Moffett Field}\\
%       \affaddr{California 94035}\\
%       \email{fogartys@amesres.org}
%% 6th. author
%\alignauthor Charles Palmer\\
%       \affaddr{Palmer Research Laboratories}\\
%       \affaddr{8600 Datapoint Drive}\\
%       \affaddr{San Antonio, Texas 78229}\\
%       \email{cpalmer@prl.com}
}
% There's nothing stopping you putting the seventh, eighth, etc.
% author on the opening page (as the 'third row') but we ask,
% for aesthetic reasons that you place these 'additional authors'
% in the \additional authors block, viz.
%\additionalauthors{Additional authors: John Smith (The Th{\o}rv{\"a}ld Group,
%email: {\texttt{jsmith@affiliation.org}}) and Julius P.~Kumquat
%(The Kumquat Consortium, email: {\texttt{jpkumquat@consortium.net}}).}
%\date{30 July 1999}
% Just remember to make sure that the TOTAL number of authors
% is the number that will appear on the first page PLUS the
% number that will appear in the \additionalauthors section.

\maketitle
\begin{abstract}
It impossible to write software that is 100 percent fault free. Faults or bugs are an inevitable aspect of computer programming. While some tools at our disposal help identify all the syntax errors at compiling, identifying the semantic errors has proven to be more challenging. Debugging for semantic errors usually is a very tedious and time consuming activity. This is mainly due to the fact that a fault can originate at different locations in the code. In this project, we aim at addressing this issue by implementing an Information Retrieval (IR) system that will analyze the source code and upon the existence of a failure, it will identify the possible locations for the fault. 
\end{abstract}


%
% The code below should be generated by the tool at
% http://dl.acm.org/ccs.cfm
% Please copy and paste the code instead of the example below. 
%

%\begin{CCSXML}
%<ccs2012>
%<concept>
%<concept_id>10010583.10010717.10010733.10010734</concept_id>
%<concept_desc>Hardware~Bug detection, localization and diagnosis</concept_desc>
%<concept_significance>500</concept_significance>
%</concept>
%<concept>
%<concept_id>10002951.10003317.10003347.10003349</concept_id>
%<concept_desc>Information systems~Document filtering</concept_desc>
%<concept_significance>300</concept_significance>
%</concept>
%<concept>
%<concept_id>10002951.10003317.10003359.10003362</concept_id>
%<concept_desc>Information systems~Retrieval effectiveness</concept_desc>
%<concept_significance>300</concept_significance>
%</concept>
%<concept>
%<concept_id>10011007.10011074.10011099</concept_id>
%<concept_desc>Software and its engineering~Software verification and validation</concept_desc>
%<concept_significance>300</concept_significance>
%</concept>
%</ccs2012>
%\end{CCSXML}
%
%\ccsdesc[500]{Hardware~Bug detection, localization and diagnosis}
%\ccsdesc[300]{Information systems~Document filtering}
%\ccsdesc[300]{Information systems~Retrieval effectiveness}
%\ccsdesc[300]{Software and its engineering~Software verification and validation}


%
% End generated code
%

%
%  Use this command to print the description
%
\printccsdesc

% We no longer use \terms command
%\terms{Theory}

\keywords{Bug localization; information retrieval; corpus creation}

\section{Introduction}
The introduction

\section{Problem}
Identifying faulty methods in a software project using bug reports as input.

\section{Background}

\subsection{IR-Based Bug Localization}
Information Retrieval (IR) can be defined as the process of finding material of an unstructured nature in a large collection to satisfy a certain information need \cite{manning2008:ir}. Even though source code is technically not unstructured text, IR-based approaches to bug localization have shown promising results over recent years \cite{zhou2012, poshyvanyk2007, saha2013}. The idea is that with some preprocessing, the source code of a certain version of a software product can be indexed and treated the same way as a large corpus of natural-language documents such as news articles or books. This allows stakeholders to query the codebase of a project using natural language and extract all kinds of information from it, provided that some conditions inherent to the particular problem domain are met. This has the notable advantage of being very easy to implement, but the main disadvantage of relying on the code containing adequately named identifiers and meaningful comments [ref?].

\subsection{Linking fixes in version history to issues in bug-tracking systems}
\cite{dallmeier2007}

\section{Implementation Plan}
We propose the implementation of a technique for bug-localization using IR. The general steps for such approach are outlined next.
First, it is necessary to index the source code in order to enable efficient search over it. For this, some preprocessing must be applied to the source files:
\begin{enumerate}
\item Split source classes into methods, and index each one as a separate document.
\item Extract comments, literal strings, …, and identifiers from each method.
\item Apply identifier splitting \cite{saha2013}.
\item Elimination of stop words and language-specific keywords.
\item Stemming.
\end{enumerate}

Once the index is formed the next step is building a query processor. In our case, the queries are bug reports posted on a bug-tracking system, and they normally consist of title and description. The same process outlined above is applied to both fields, and an information retrieval system such as lucene is used to rank the documents in order of descending relevance.

We will also implement a functionality to extract the gold set from a repository \cite{dallmeier2007}.

\section{Experimental Design}
We will collect a gold set from at least 10 open source software projects. We will apply the technique splitting each query in title only, description only and both title and description. We will compare the performance on each system with IR metrics \cite{saha2013} and draw some conclusions by manually analyzing some of the results.

%
% The following two commands are all you need in the
% initial runs of your .tex file to
% produce the bibliography for the citations in your paper.
\bibliographystyle{abbrv}
\bibliography{project}  % sigproc.bib is the name of the Bibliography in this case
% You must have a proper ".bib" file
%  and remember to run:
% latex bibtex latex latex
% to resolve all references
%
% ACM needs 'a single self-contained file'!
\balancecolumns

\end{document}

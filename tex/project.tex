\documentclass{sig-alternate-05-2015}

\begin{document}

\title{Software Testing, Validation and Verification Class Project Mid-Term Report}

\numberofauthors{1}

\author{
\alignauthor
Juan Manuel Florez Fandino, Raul Quinonez Tirado \\ 
       \affaddr{University of Texas at Dallas}\\
       \affaddr{800 W. Campbell Road}\\
       \affaddr{Richardson, TX}\\
       \email{\{jflorez, Rxq100020\}@utdallas.edu}
}

\maketitle
\begin{abstract}
Faults or bugs are an inevitable aspect of computer programming. Bug reports are normally used to let developers know 
there is an error in the code, but they rarely provide all the information necessary to determine where the bug must be 
fixed at first glance. In this project, we aim at addressing this issue by implementing an Information Retrieval (IR) 
system that will leverage the information found in bug reports to identify the possible locations in the source code 
where the fault is located.
\end{abstract}

%
% The code below should be generated by the tool at
% http://dl.acm.org/ccs.cfm
% Please copy and paste the code instead of the example below. 
%

%\begin{CCSXML}
%<ccs2012>
%<concept>
%<concept_id>10010583.10010717.10010733.10010734</concept_id>
%<concept_desc>Hardware~Bug detection, localization and diagnosis</concept_desc>
%<concept_significance>500</concept_significance>
%</concept>
%<concept>
%<concept_id>10002951.10003317.10003347.10003349</concept_id>
%<concept_desc>Information systems~Document filtering</concept_desc>
%<concept_significance>300</concept_significance>
%</concept>
%<concept>
%<concept_id>10002951.10003317.10003359.10003362</concept_id>
%<concept_desc>Information systems~Retrieval effectiveness</concept_desc>
%<concept_significance>300</concept_significance>
%</concept>
%<concept>
%<concept_id>10011007.10011074.10011099</concept_id>
%<concept_desc>Software and its engineering~Software verification and validation</concept_desc>
%<concept_significance>300</concept_significance>
%</concept>
%</ccs2012>
%\end{CCSXML}
%
%\ccsdesc[500]{Hardware~Bug detection, localization and diagnosis}
%\ccsdesc[300]{Information systems~Document filtering}
%\ccsdesc[300]{Information systems~Retrieval effectiveness}
%\ccsdesc[300]{Software and its engineering~Software verification and validation}


%
% End generated code
%

%
%  Use this command to print the description
%
\printccsdesc

\keywords{Bug localization; information retrieval; corpus creation}

\section{Introduction}
Testing software products is one of the most important tasks in the software development cycle. Throughout the years, 
several approaches have been developed in order to make this task easier and more effective. While most of the time 
it is infeasible to do exhaustive testing on any piece of software, testing is usually carried out with the goal 
of improving our confidence on the correctness of the code. Software that contains many faults is usually 
considered of low-quality and undesirable and the goal of testing is to identify the largest number of faults 
before the code is ready for release. Once software products are released, they need to be maintained for a 
defined period of time. This highlights the importance of constantly debugging a piece of software. Several 
techniques have been proposed in order to ease the debugging process. Information Retrieval is one of the 
techniques used to localize bugs.

The main goal of IR-based bug location systems is to locate bugs by taking as input a textual bug report and 
a collection of source code files \cite{le2014}. The system then will output a ranked list of files of the 
possible location of the bug. Developers will then look at the highest ranked locations for the bug first. 
One problem that might arise is that sometimes the real cause for bugs is not correctly identified at the top of 
the ranked list. With our implementation, we aim at improving the effectiveness of bug localization.

\section{Problem}
Identifying faulty files in a software project using bug reports as input.

\section{Background}

\subsection{IR-Based Bug Localization}
Information Retrieval (IR) can be defined as the process of finding material of an unstructured nature in a large 
collection to satisfy a certain information need \cite{manning2008}. The most widespread application of this discipline 
is internet search engines. These allow efficient search over millions of different web pages with the information need 
being presented in the form of a natural-language query by a user. IR usually operates on a \textit{corpus}, 
i.e. a collection of documents which can be natural-language text such as books or news articles, but also 
multimedia like images or videos, or any combination of these. The typical output of such a system is a ranked 
subset of the corpus which hopefully contains the documents that are most relevant to the user's information 
need among the first few entries.

Even though source code is technically not unstructured text, IR-based approaches to bug localization have shown 
promising results over recent years \cite{zhou2012, poshyvanyk2007, saha2013, rao2011, dit2013}. The idea is 
that with some preprocessing, the source code of a certain version of a software product can be indexed and given 
the same treatment as a corpus of natural-language documents. This allows stakeholders to query the codebase of 
a project using natural language and extract all kinds of information from it. This has the notable advantage of 
being very easy to implement, but the main disadvantage of relying on the code containing adequately named 
identifiers and meaningful comments \cite{saha2014}.

% Explain VSM and why.

\subsection{Linking fixes in version history to issues in bug-tracking systems}
IR approaches normally have to be empirically tested in order to assess their performance. This process is 
normally carried out using special corpora for which some queries have been manually audited and all 
(or a good part) of the relevant results for each one are known \cite{harman2011}. This also applies to the 
field of automatic bug localization, in which we talk about a \textit{gold set}, i.e. a set of bug reports 
connected to the related program elements that were modified to address each one. Even though some approaches 
to automate this process have been proposed \cite{dallmeier2007}, they lack the precision to construct a 
reasonably reliable gold set.

The method that most of these approaches propose is \cite{sliwerski2005}:
\begin{enumerate}
	\item Begin with the change history of a software project and the bug-tracking system for the project.
	A bug tracker is not necessary but it help increase the confidence in extracted candidate bug fixes.
	
	\item Examine the commit messages for the software release being considered and extract anything resembling
	an issue identifier. The specific format of these depends on the specific bug-tracking system used by the
	project, and could be simply a number, a URL or a string like \emph{PROJECT-0000} as is the case with
	Apache projects.
	
	\item Increase the confidence of the candidate bug ids extracted by querying the version control system.
	If the issue is marked as fixed or assigned to a developer, the confidence level for the candidate bug
	fix increases. Similarly,
	if the issue contains attachments referencing the files being committed, confidence increases.
\end{enumerate}

The main problem of this kind of approach is that they report a considerable number of false positives, since
potentially any number included in a commit message for other reasons could be interpreted as an issue id
if it exists in the bug-tracking system by chance. Additionally, not all fixed bugs get marked as such in these
systems, and some of them get fixed without explicitly being assigned to a developer.

\section{Implementation Plan}
We propose the reimplementation for bug-localization tool known as BugLocator \cite{zhou2012}. This tool uses IR
methods to locate bugs in a codebase using bug reports as input. The tool also leverages document length and
previous recorded bug fixes to improve the results. The idea behind this is that long code files are more likely
to contain bugs than shorter files, and files that have been fixed before are similarly more likely to contain
defects than those that haven't been modified as much. The technique uses file-level granularity, i.e. given
a bug report, the result will be a ranked list of source files.

As it is true for most IR applications, it is necessary to index the source code in order to enable efficient 
search over it. For this, some preprocessing must be applied to the source files:

\begin{enumerate}
	\item First, the correct versions for the software project under study must be selected. The gold set will be
	collected for one or more versions of the source code. Since it only makes sense to look for a bug in the version
	in which it was reported, these different versions must be identified and used separately.
	
	\item Since the chosen granularity chosen for this application is file-level, the following step is iterating through
	the project source files and applying the next steps.
	
	\item Extract comments, literal strings and identifiers from each file. A parser for the particular programming
	language is not mandatory but will help identify these elements unambiguously. It is necessary to ignore reserved 
	words of the language and programming tokens such as \texttt{\{} since these don't contain any semantic information
	that would boost search performance.
	
	\item Apply identifier splitting. This process consists in recognizing composite identifiers such as 
	\texttt{threadExecutor} and splitting them into their constituent words, in this case \texttt{thread} and 
	\texttt{executor}. The original identifier is kept as well, since this has been shown to improve search effectiveness
	\cite{saha2013}.
	
	\item Eliminate stop words. These are words that don't contain meaningful information from the perspective of IR, 
	such as conjunctions, pronouns and articles (\textit{for}, \textit{about}, \textit{it}, etc.) This is because they
	appear in most documents, meaning they don't contain a lot of information from the perspective of information theory.
	
	\item Apply stemming. This is a process that reduces words to their base form, e.g. ``implementing'' and 
	``implementation'' get 	reduced to ``implement''. This helps reducing the amount of possible terms for searching and
	has been shown to increase performance \cite{saha2013}. The Porter stemmer is a popular choice for this process 
	\cite{porter1980}.
	
	\item An IR tool such as Lucene\footnote{https://lucene.apache.org/} is then used to store the processed representation
	of each source file to allow efficient search over the whole corpus.
\end{enumerate}

Once the index is formed the next step is building a query processor. In our case, the queries are bug reports posted 
on a bug-tracking system, and they normally consist of title and description. The same process outlined above, excluding 
the first three steps, is applied to both fields and an information retrieval system such as Lucene is used to rank the 
documents in order of descending relevance.

Relevance calculation is the step that will really determine the effectiveness of our implementation. Three different
attributes will be used for this process: textual similarity, file length and similar bugs.

\subsection{Textual Similarity}
In the VSM model, a source file is represented as a vector in which each dimension represents a term in the corpus, and
its value is the amount of times that term appears in the document. For example, if we have a corpus consisting of the
terms $\{\text{black}, \text{funny}, \text{sheep}, \text{cat}\}$ and a document $d=\{\text{black}, \text{cat}, \text{funny},
\text{cat}\}$, the vector representation for this document will be $V_d = (1, 1, 0, 2)$. When comparing two vector
representations, in our case those referring to a bug report and a source file, simply calculating the distance
between the two points might not be the best approach \cite{manning2008}. A technique called \textit{tf-idf} is
widely used in the IR field, and it consists on calculating the angle between the two vectors. This helps reduce the
bias towards longer documents, which is a big problem when using only euclidean distance.

\subsection{File length}

\subsection{Similar bugs}

\section{Experimental Design}
After the IR approach is implemented, it will be empirically tested on at least 5 open source software projects. 
For this, a gold set will be manually collected both from existing data and by making use of version history 
repositories and bug-tracking systems.

The implemented IR approach will be tested by performing search over the indexed source code using each query 
(bug report) differentiating between the results obtained using only the title of the bug report as a query, using 
only the description and using both. The results will be evaluated using some widely used IR metrics 
\cite{saha2013, manning2008}. With these results, a comprehensive analysis of the strengths and weaknesses of 
the approach will be carried out, which will allow us to propose candidates for future work.

\bibliographystyle{abbrv}
\bibliography{project}
\balancecolumns

\end{document}
